\chapter{Abstract}\label{kapitel_Abstract}
\textbf{Background}
All Danish women between the age of 50 to 69 years are offered 
mammography to screen for breast cancer. This method of screening may sometimes be unsuitable, as it is hard to distinguish between glandular tissue and tumors in an x-ray image - and some women have a lot of glandular tissue. In these cases it is necessary to additionally perform an ultrasonography procedure.

Nowadays, mammography procedures are carried out by radiographers or radiology nurses, after which the x-ray images are sent to a radiologist. The radiologist then decides if additional examinations are required. It is possible to envision that automatic ultrasonographic procedures could be accomplished with the same work flow as mammography procedures.

This bachelor thesis therefore concerns the development of the system Automatic Sonography and the possibility of performing automated ultrasonographic procedures to compliment the program for breast cancer screening.

\textbf{Methods}
Elements of Scrum was used to organize the project in the development process. A user survey, a medical approval draft and an economic analysis have been produced to investigate which approaches are needed to realize Automatic Sonography. UML and SysML have been used to describe Automatic Sonography.

\textbf{Results}
A system with the name Automatic Sonography has been developed. Automatic Sonography consists of a Windows application, which makes it possible to receive a 3D model of a chest, which a robot arm can move along. It has not been possible to instruct the robot arm to move in a specific movement pattern which is necessary to perform ultrasonography.

\textbf{Conclusion}
A system, which partially meets the requirements specified according to the delimitation, has been developed. The possibility of securing a medical approval of Automatic Sonography has been investigated, however, the system was not medically approved. The benefit of using a combination of ultrasound and x-ray is the discovery of earlier stages of cancer, and thus the survival rate of patients with cancer would be higher. A disadvantage is the overdiagnosis associated with screening. The addition of ultrasonographic procedures to the screening program might not necesarily be cost effective.