\chapter{Diskussion}\label{kapitel_Diskussion}
Med Automatisk Ultralydsscanner er det opnået at lave et system, der kan lave et 3D billede og derefter få Robotarm til at rotere omkring testobjektet. Alle funktionelle og ikke-funktionelle krav på nær hovedscenariet i UC3: Ultralydsscan brystområdet blev godkendt i accepttesten. 

Det var ikke muligt for Robotarm at udføre bevægelser i det specifikke bevægelsesmønster, beskrevet i Bilag \ref{Kravspecifikation} Kravspecifikation, hvilket gjorde, at UC3: Hovedscenarie fejlede accepttesten. Det betyder ikke, at Robotarm ikke kan bevæge sig over det område som 3D kamera har detekteret, men at det ikke har været muligt at implementere det specifikke sinuslignede bevægelsesmønster. Robotarmens bane over brystet er begrænset af 3D kameras detektering af overfladen, hvor der kan opstå ujævnheder. Det er i PC applikationen forsøgt at lave en gennemsnitlig overflade med en laplace udglatningsfunktion for at undgå ujævnheder, men 3D kameraet er upræcist, hvilket resulterer i, at de punkter Robotarm benytter som pejlemærker også er upræcise. Det vil være nødvendigt at anvende et mere præcist kamera eller benytte et kamera, der kan monteres på robotarmen for at få billeder fra flere vinkler og derfor en bedre 3D model. 

For at lave rotationen af det yderste led, er der lavet pitch, yaw og roll beregninger, som er implementeret i PC applikation. Robotarmens yderste led roterer, hvilket det ikke bør, og det betyder, at Robotarm i visse tilfælde sidder fast grundet rotationen. Det formodes, at problemet kan løses ved at anvende (!!MATHIAS) 

EVT ANDET SOM SKAL MED FOR AT SIGE HVAD DER GIK GALT 


Det var nødvendigt at give operatøren mulighed for at kunne se og godkende billedet fra 3D kameraet, da der er tilfælde, hvor billedet er deformt. Det blev derfor valgt at integrere outputtet fra Microsoft Kinect 3D kameraet, og dennes tilhørende API. Derfor blev det også muligt at integrere en funktionaliteten, der kan begrænse område, man ønsker at 3D scanne. Denne udvidelse til UC2: 3D scan brystområde blev valgt for at undgå at få f.eks. hager med på patienter. Funktionaliteten giver patienten mulighed for at ligge i en behagelig stilling på briksen, og operatøren har større rådighed over, hvor der præcist skal scannes. 

I projektet er performancetiderne til systemet blevet prioriteret højt, da det potentielt vil kunne forbedre mulighederne for implementering af systemet i sundhedsvæsenet, hvor tid er en vigtig faktor. Til accepttesten tog det systemet omkring 2 minutter for at afvikle først 3D scanning og derefter rotere Robotarm rundt på området, hvor det i kravspecifikationen bilag \ref{Kravspecifikation} er et krav, at det må tage 10 minutter til sammen. En 3D scanning er sat til at må tage 10 sekunder, men til accepttesten tog det omkring 2 sekunder at vise 3D billedet på GUI. PC Applikation brugte under 1 sekund i stedet for de 10 sekunder, for at færdiggøre brystområdets positurer til Robotarm. Performancetiderne lå generelt et godt stykke under kravene defineret ud fra interviews med radiolog Lars Bolvig og afdelingsradiograf Tine Bisgaard, men for at kunne tages i brug, er der flere faktorer. 

Der er mangel på radiologer i Danmark, såvel som resten af verden \ref{Lagemangel}, hvorfor man skal tænke andre metoder og arbejdsgange. På længere sigt kan man forestille sig, at radiografer overtager nogle af opgaverne fra radiologerne bl.a. ultralydsscanninger. Ved f.eks. at lade radiografer lave selve scanningerne, vil patienter vil først få besked efter radiologen har tilset scanningerne, men den procedure er også brugt til mammografi med røntgen. Det kan diskuteres, om en udvidelse af screeningsprogrammet vil være en god ting. Udvidelsen vil give mulighed for at finde flere kræfttilfælde, men der er risiko for overbehandling og øgede udgifter til sundhedsvæsenet. Man kunne i stedet forestille sig, at en del af røntgen scanningerne kunne erstattes af ultralydsscanninger, da 1 ud af 100.000 patienter udvikler kræft røntgenstråling, mens der ikke er nogen kendte bivirkninger ved ultralydsscanninger af brystet. Røntgenundersøgelser er billigere end ultralydsundersøgelser, bl.a. fordi at en radiolog laver scanningerne. 

Den medicinske godkendelse burde have været implementeret i kravspecifikationens funktionelle krav til Automatisk Ultralydsscanner for at overholde lovgivning til CE-mærkning. Det blev besluttet at prioritere udviklingen af Automatisk Ultralydsscanner højere end den medicinske godkendelse, da bachelorprojektet skulle undersøge muligheden for at udvikle et system til automatiseret ultralydsscanning af mamma mhp. screening for brystkræft. Den medicinske godkendelse vil derfor kunne have været en forhindring i denne udviklingsproces, og andre krav ville ikke være blevet implementeret. 

Hvis et system til automatisk ultralydsscanning af brystet skal medicinsk godkendes er det vigtigt først at tjekke, om hver komponent er godkendt til medicinsk brug, for at kunne bruge det i kombination med hele systemet. At få en medicinsk godkendelse kan tage mange år, og UR10 er ikke godkendt til medicinsk brug. Derfor kan det være bedre at udvikle systemet til automatisk ultralydsscanning på en robot, der allerede er godkendt til medicinsk brug. 

