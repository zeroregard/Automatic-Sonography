\chapter{Diskussion}\label{kapitel_Diskussion}
Det er vigtigt at udpege og diskutere relevante dele af de opnåede resultater og deres betydning. Der skal også gives en samlet vurdering af de opnåede resultater med relation til problemstillingen og formålet med – eller hypotesen for – projektet. Der kan ligeledes være en opsummerende beskrivelse af resultater I er særligt stolte af. Diskussion af resultater fylder typisk 2 sider.

Er problemstillingen løst? 

Metoder
Økonomi
Den økonomiske analyse er unøjagtig, da alle de anvendte tal er et skøn og de vurdering af, hvor lang tid en scanning af brystet tager, er lavet ud fra en radiologers udtalelse. Generelt er det forsøgt at sætte udgifterne forbundet med indførslen af den Automatiske Ultralydsscanner relativt højt for at imødekomme uforudsete omkostninger. Derudover kan det diskuteres, om priserne for opsætning vil være lavere, hvis man laver en indkøbsaftale. Det ses tydeligt, at det er ved transporttiden, at der kan spares, men det er ikke beregnet med, at radiologen vil skulle scanne flere på en gang, efter at have kørt for eksempel 60 minutter til et sted. Transporttiden er valgt som en variabel efter et interview med overlæge Lars Boldvig, der tydeliggjorde, at transporttid var noget, han brugte lang tid på. Priser for vedligeholdelse af ultralydsscanneren er ikke beregnet med, da dette vil være ens for begge scenarier. Priser for generel vedligeholdelse er dog hellere ikke beregnet med andet end serviceaftalen, da det vil være individuelt fra hospital/afdeling/praksis mm. om de vil have oplært en pedel eller lignende. Årligt bliver der foretaget omkring 23.000 ultralydsundersøgelse af brystet i Region Midtjylland, og omkring 127.000 i hele landet. Det har ikke været muligt at finde præcise tal på, hvor mange af disse undersøgelser er scanninger eller screeninger, men i analysen er det beregnet at størstedelen er ultralydsscanninger foretaget af en radiolog.  

Brugerundersøgelse
Da spørgeskemaundersøgelsen blev lagt ud på de sociale medier, kan man formode at de adspurgte er venner af bachelorgruppens medlemmer, hvilket kan gøre, at de adspurgte kan være påvirket til at svare positivt på spørgeskemaet, samtidig kan de adspurgte have hørt en af gruppemedlemmerne, tale godt for en automatiseret robotarm til ultralyd. De adspurgte kan også være klassekammerater til gruppemedlemmerne og derved studerende på ingeniørhøjskolen, hvilket kan gøre at de svare positivt i forhold til teknologi. Spørgeskemaundersøgelsen er derfor ikke repræsentativ. Spørgeskemaundersøgelsen er dog alligevel medtaget i projektet, da især ulemper og de adspurgtes bekymringer kan imødekommes med oplysning og information om hvordan Automatisk Ultralydsscanner fungerer. 

Medicinsk godkendelse
Den medicinske godkendelse burde have været implementeret i kravspecifikationen til Automatisk Ultralydsscanner, men den medicinske godkendelse blev først udarbejde sent i udviklingsprocessen og er derfor ikke anvendt til udviklingen af Automatiks Ultralydsscanner. Hvis der havde været kendskab til krav fra MDD før udviklingen af Automatisk Ultralydsscanner startede, kunne disse krav have været tænkt med i udviklingen og implementeret i systemet. Generelt for alle standarder, der er fulgt, vil det have været lettere at følge dem i udviklingen af Automatisk Ultralydsscanner, hvis man kendte til dem inden udviklingsprocessen startede. Dog overholder Automatisk Ultralydsscanner alligevel lovgivningen inden for nogle områder, for eksempel er softwaren testet, dokumenteret, gjort mulig at opdatere, men indeholder endnu ikke en procedure for hvad brugeren skal gøre, hvis der sker fejl. 

Løsning
Der er fokuseret på de funktionelle krav.
Use cases?
Robottens bevægelser. 
Der er ikke foretaget brugertest. 

En tryksensor kunne have forbedre Automatisk Ultralydsscanner, da den nuværende version ikke vil stoppe Robotarm hvis den trykker for hårdt, altså hvis trykket kommer over den tærskelværdi der er givet. Samtidig vil Robotarm heller ikke tage højde for hvis den slet ikke leverer noget tryk. Dette vil en tryksensor og yderligere software kunne korrigere for. 

Som ultralydsscanninger foretages i dag, kan patienten få besked af radiografen med det samme, om der er behov for videre undersøgelser. Dette kan man ikke ved brug af Automatisk Ultralydsscanner, da radiografen først skal tilse billederne. Da proceduren ikke er forskellig fra proceduren ved mammografi ses der umiddelbart ingen problemer ved dette. 

Værktøjer/udstyr/proces
kinect optimalt?
- scanningerne kan ikke reproduceres fuldstændigt, da kinect vil tage et nyt billede ved hver scanning. 
- Først var planen at kinect skulle sidde på robotarmen, så der kunne tages billeder fra flere vinkler. Dette blev dog droppet, da kinect var for stor og derfor ville sidde i vejen for ultralydsproben. 

Test
blev den gennemført som forvente.
Testen blev udført på et 3D printet bryst, hvilket ikke er specielt virkelighedsnært.  
 




Accepttest vil være bedre, hvis der var radiolog og ultralydsscanner tilrådighed. Radiolog til at tjekke og sikre, at robottens bevægelser er korrekte. 

