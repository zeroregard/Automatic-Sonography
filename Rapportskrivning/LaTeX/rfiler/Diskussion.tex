\chapter{Diskussion}\label{kapitel_Diskussion}
Med Automatisk Ultralydsscanner er det opnået at lave et system, der kan lave et 3D billede af et testobjekt og derefter få Robotarm til at rotere omkring dette. Alle funktionelle og ikke-funktionelle krav på nær hovedscenariet i UC3: Ultralydsscan brystområdet blev godkendt i accepttesten: Det betyder ikke, at Robotarm ikke kan bevæge sig over det område, som 3D kamera har detekteret, men at det ikke har været muligt at implementere krav til bevægelsesmønsteret, som kan ses på side \pageref{Probensbevagelse}. Robotarms bane over brystet er begrænset af 3D kameras detektering af overfladen. I detekteringen kan der opstå ujævnheder, da 3D kamera er upræcist. Derfor er det forsøgt at lave en gennemsnitlig overflade med en udglatningsfunktion for at undgå ujævnheder. Udglatningsfunktionen, der er anvendt, gør, at der kan opstå problemer ifm. bestemmelse af stien, Robotarm skal dække. Det kan derfor diskuteres, om et mere præcist 3D kamera vil hjælpe, om der skal benyttes bedre udglatningsfunktioner, eller om det er en kombination af begge, der vil kunne forbedre stien. 

Beregningen af Robotarms rotation er implementeret i PC Applikation. En bivirkning af beregningen er, at Robotarms yderste led roterer. Dette er uønsket af to grunde: En radiolog vil aldrig rotere en ultralydsprobe på denne måde, og det betyder, at Robotarm kan over- eller underrotere dette led. Dette resulterer i, at Robotarm kan stoppe op i løbet af ultralydsscanningen. Det blev besluttet, at delvist rotering af Robotarm mod brystets overflade er vigtigere, end at Robotarm trykker med den korrekte kraft på en plan overflade. Det blev vurderet, at det er nemmere at implementere tryk-korrigering, såfremt rotationen  er implementeret. Derudover havde et tidligere bachelorprojekt beskrevet problematikken bag rotering af UR10, og derfor gav det også anledning til at løse dette problem.

Operatør har mulighed for at stoppe ultralydsscanningen ved en knap på GUI'en. Robotarm stopper ikke brat, da den først bremser ved den næste positur i ultralydsscanningsstien. Vigtigheden for, hvor hurtigt Robotarm stopper fra et kald i GUI'en, kan overvejes, da Robotarm er forsynet med dens egen nødstop-knap, hvor Robotarm vil stoppe brat ved tryk på denne knap.

Til accepttesten var performance-tiderne relativt lavere end kravene specificeret i bilag \ref{Kravspecifikation} Kravspecifikation. Performance-tiderne er meget afhængige af de beslutninger, der tages for, hvor høj kvaliteten af ultralydsscanningen skal være. En højere opløsning på 3D scanningen samt flere gennemgange af udglatningsfunktionen vil øge den tid, PC Applikation bruger på at finde positurer. Det kan overvejes, hvordan dette vægtes bedst muligt.   
\newpage
Interviews med Lars Bolvig og Tine Bisgaard har været med til at definere systemets krav. Validiteten kan diskuteres, da der kan opstå bekræftelsesbias. Vigtigheden af spørgeskemaundersøgelsen kan diskuteres, da den ikke er brugt til udviklingen af systemet. Intentionen var at give en indikation af modtagelsen af en Automatisk Ultralydsscanner, hvilket var overvejende positivt. I en fremtidig udgave kan det forestilles, at det vil være bedre at foretage brugertests i stedet for spørgeskemaer, for at udvikle et system, brugere vil være tilfredse med at benytte. 

Der er mangel på radiologer i Danmark, såvel som resten af verden \cite{Lagemangel}, hvorfor man skal tænke andre metoder og arbejdsgange. På længere sigt kan man forestille sig, at radiografer overtager nogle af opgaverne fra radiologerne, heriblandt bl.a. ultralydsscanninger. Man kunne overveje, hvorfor en radiograf ikke blot kunne udføre det arbejde, som Robotarm gør. Selv med de yderligere implementeringer nævnt i afsnit \ref{kapitel_Fremtidig udvikling} Fremtidig udvikling, bør der måske anvendes machine learning for at få Automatisk Ultralydsscanners ultralydsscanningskvalitet på samme højde som en radiologs. Altså vil det give mere mening at uddanne en radiograf i at udføre ultralydsscanninger, såfremt at systemet ikke er sikret at levere en scanning på niveau med en professionel radiolog.

Det kan diskuteres, om en udvidelse af screeningsprogrammet vil være en god ting: Udvidelsen vil give mulighed for at finde flere kræfttilfælde - men der er risiko for overbehandling og øgede udgifter til sundhedsvæsenet. Man kunne i stedet forestille sig, at en del af røntgenscanningerne kunne erstattes af ultralydsscanninger, da 1 ud af 100.000 \cite{Risk} patienter udvikler kræft af røntgenstråling, mens der ikke er nogen kendte bivirkninger ved ultralydsscanninger \cite{Ultralydsscanning}. 

Den medicinske godkendelse burde have været implementeret i kravspecifikationens funktionelle krav til Automatisk Ultralydsscanner, for at overholde lovgivning til CE-mærkning. Det blev besluttet at prioritere udviklingen af Automatisk Ultralydsscanner højere end den medicinske godkendelse, da bachelorprojektet skulle udføre Proof of Concept af et system til automatiseret ultralydsscanning af mamma mhp. screening for brystkræft. Det kan tage mange år at få udstyr godkendt til medicinsk brug, og derfor er det en stor fordel at anvende allerede godkendte komponenter. UR10 er f.eks ikke godkendt til medicinsk brug. hvilket kan ses bilag  \ref{UserManualUR10} User Manual UR10, hvorfor det vil være nødvendigt at udskifte UR10 med en godkendt Robotarm. 


 
