\chapter{Forord}
Dette bachelorprojekt er udarbejdet af Marie Kirkegaard, Charlotte Søgaard Kristensen og Mathias Siig Nørregaard og er det afsluttende projekt på 7. semester for diplomingeniøruddannelsen i henholdsvis Sundhedsteknologi og Informations- og Kommunikationsteknologi ved Aarhus University School of Engineering (ASE). Bachelorprojektet er udarbejdet i perioden 29. august 2016 til 16. december 2016, og ideen bag bachelorprojektet er opstået hos Søren Holm Pallesen, stifter og direktør for Robotic Ultrasound. 

Formålet med projektet er at undersøge muligheden for at lave automatiske og reproducerbare ultralydscanninger til screening for brystkræft. Bachelorprojektet henvender sig primært til ingeniører indenfor Sundhedsteknologi samt Informations- og  Kommunikationsteknologi. 

Bachelorgruppen vil gerne rette en stor tak til alle som har hjulpet med sparring gennem hele projektets forløb:

\let\labelitemi\labelitemii
\begin{itemize}
\item Associate Professor Michael Alrøe for support og engageret vejledning gennem projektforløbet.

\item Cand.scient.med. Søren Holm Pallesen for at stå til rådighed ved diverse relevante møder, og for altid at have gode ideer til projektarbejdet.

\item Radiolog og ultralydsekspert Lars Bolvig Hansen for telefoninterview og teknisk sparring på ultralydsområdet.

\item Afdelingsradiograf Tine Bisgaard for besøg og rundvisning på Røntgen- og Skanningsafdelingen på Aarhus Universitetshospital, Tage Hansen Gade. 
\end{itemize}

\section{Læsevejledning} 
Bachelorprojektet er delt op i en projektrapport og procesrapport. Projektrapporten beskriver selve projektet, hvordan problemformuleringen er løst, samt hvilke resultater der er opnået. Procesrapporten, som findes efter projektrapporten, beskriver udviklingsprocessen af Automatisk Ultralydsscanner. Dette inkluderer projektadministrationen, arbejdsfordeling, planlægning og brug af projektstyringsværktøjer.

Der skelnes mellem navneord med stort og småt. F.eks robotarm og Robotarm, hvor Robotarm er aktøren i systemet og robotarm henviser til en vilkårlig robotarm.
I hvert afsnit er der refereret til bilag, hvor man kan finde uddybende dokumentation. Bilag til specikke dokumenter findes nederst i hver rapport, mens den fulde oversigt over alle bilag kan findes i bilag \ref{Bilagsliste} Bilagsliste.  
