\chapter{Afgrænsning}
\label{Afgransning}

I udviklingsforløbet blev det besluttet ikke at inkludere eksterne sensorer eller software til målingen af påført tryk, og dermed har det ikke være nødvendigt at have en ultralydsscanner til rådighed. Det er i stedet valgt at prioritere 3D kameraets genkendelse af dybde i brystområdet, og derefter bruge billedet til at få robotarmen til at bevæge sig rundt på det detekterede område. Det blev bedømt at disse elementer var vigtigere at implementere, for at udføre et Proof of Concept.

Der er i projektet blevet benyttet et 3D kamera af typen Kinect 2.0, som er et forholdsvis et stort kamera. Det blev vurderet at det ville være besværligt at have både en ultralydsprobe og en Kinect monteret på robotarmen på samme tid. Det er derfor valgt at montere 3D kameraet i loftet. 

Armhulerne scannes også ved en konventionel ultralydsscanning af brystet. Det er valgt at afgrænse til scanning af brystet, da armhulerne besværgerliggører operatørs beskærimg af 3D scanningen. Scanning af armhuler vil resultere i en 3D model af uinteressante områder, hvorfor brystområdet er valgt som afgrænsning.  
