\chapter{Medicinsk godkendelse}\label{MedicinskGodkendelse}
Den medicinske godkendelse er lavet for at undersøge, hvad der skal til, for at Automatisk Ultralydsscanner kan blive CE-mærket og derved godkendt til markedsføring i Europa. 

Medical Device Directive 93/42/EØF (MDD)\cite{MDD} er hoveddirektivet for medicinsk udstyr i Europa og danner grundlag for de godkendelsesprocedurer, der skal til, for at få medicinsk udstyr CE-mærket. Det er et lovkrav at overholde MDD, når man godkender medicinsk udstyr. Alt kursiv i dette afsnit, er citater fra MDD.

\section{Definition}
Medicinsk udstyr er i MDD defineret som: 

\emph{»Medicinsk udstyr«: Ethvert instrument, apparat, udstyr, software, materiale eller anden genstand anvendt alene eller i kombination, herunder software, som af fabrikanten er beregnet til specifik anvendelse til diagnostiske eller terapeutiske formål, og som hører med til korrekt brug heraf, og som af fabrikanten er beregnet til anvendelse på mennesker med
henblik på:}

\let\labelitemi\labelitemii
\emph{\begin{itemize}
\item diagnosticering, forebyggelse, overvågning, behandling eller lindring af sygdomme,
\item diagnosticering, overvågning, behandling, lindring af eller kompensation for skader eller handicap,
\item undersøgelse, udskiftning eller ændring af anatomien eller en fysiologisk proces, eller
\item svangerskabsforebyggelse,
\end{itemize}}

\emph{og hvis forventede hovedvirkning i eller på det menneskelige legeme ikke fremkaldes ad
farmakologisk, immunologisk eller metabolisk vej, men hvis virkning kan understøttes ad denne vej.}

Med udgangspunkt i denne definition af medicinsk udstyr, går Automatisk Ultralydsscanner under kategorien som værende medicinsk udstyr. Dette begrundes med at Automatisk Ultralydsscanners primære opgave er automatiske ultralydsscanninger til srceening for brystkræft, og derved har til formål at forebygge af sygdomme. MDD skal derfor overholdes. 

\section{Klassificering}
Da Automatisk Ultralydsscanner er medicinsk udstyr, skal der foretages en klassificering af systemet. Klassificeringen foretages for at finde ud af hvilken procedure der skal anvendes for at få CE-mærket Automatisk Ultralydsscanner. Klassificeringen afspejler den risiko, der er forbundet med anvendelsen af udstyret. Jo højere klassificering, jo højere risiko er der ved anvendelsen af udstyret og jo længere er godkendelsesproceduren for CE-mærkningen. 

Klassificeringen er som følgende: 

\item Klasse I - 
\item Klasse Is - Sterilt klasse I udstyr
\item Klasse Im - Klasse I Udstyr med målefunktion
\item Klasse IIa - 
\item Klasse IIb - 
\item Klasse III - 



I MDD, er der 18 regler, man klassificerer medicinsk udstyr ud fra. 

Regel 10 omhandler aktive anordninger beregnet til diagnosticering og overvågning af vitale fysiologiske processer. Dette passer på Automatisk Ultralydsscanner, da det er et system, som er tilsluttet en ultralydsscanner, hvilket gør Automatisk Ultralydsscanner til en aktiv anordning.  

Citat fra MDD regel 10. 
\emph{Aktiv anordninger, der er beregner til diagnosticering, henhører under klasse IIa: - hvis de er beregnet til at muliggøre en direkte diagnosticering eller overvågning af vitale fysiologiske processer...}

Da Automatisk Ultralydsscanner er beregnet til overvågning af vitale fysiologiske processer er Automatisk Ultralydsscanner klasse IIa. 

\section{CE-Mærkning}
Klassificering af Automatisk Ultralydsscanner danner grundlag for proceduren for CE-mærkning. 

Inden Automatisk Ultralydsscanner kan CE-mærkes, skal producenten igennem en række godkendelsesprocedurer. Definering og klassificering, som er gjort overfor, er en del af de procedure, producenten skal udføre. Derudover skal producenten overholde væsentlige krav fra DMD. Der skal udarbejdes teknisk dokumentation for produktet, bestående af en risikoanalyse og klinisk evaluering. Producenten skal ydermere lave et kvalitetssikringssystem og have et post market surveillance system, et system for hvordan producenten vil holde øje med produktet og andre lignende produkter, når det er kommet ud på markedet. Derudover skal producenten have en EF-overensstemmelseserklæring for, at produktet opfylder bekendtgørelsens krav. Når EF-overensstemmelseserklæring er underskrevet, kan producenten påføre CE-mærket. Som producent i Danmark, skal man registreres hos Lægemiddelstyrelsen, før markedsføringen kan påbegyndes. Producenten kan selv vælge et bemyndiget organ, som godkender, at producentens dokumentation lever op til gældende lovgivning. \cite{Klasse} 

Godkendelsesproceduren er et stort arbejde, da MDD er kompliceret at læse og forstå. Godkendelsesproceduren kan gøres lettere ved i stedet at følge en række standarder, som er harmoniseret i forhold til MDD. Til den medicinske godkendelse af Automatisk Ultralydsscanner er de harmoniserede standarder til risikohåndteringen DS/EN ISO 14971:2012 \cite{14971} og kvalitetssikring DS/EN ISO 13485:2012 \cite{13485} blevet anvendt. 

\subsection{Risikohåndtering}
Da projektet er et undersøgelsesprojekt og udviklet for at teste muligheden for at udføre automatiserede ultralydsscanninger til screening for brystkræft. Og at udviklingsforløbet startede før der var kendskab til kravet om en risikohåndtering, er Automatisk Ultralydsscanner ikke udviklet med hensyn til risikohåndteringen. Der er dog stadig udført risikohåndtering på Automatisk Ultralydsscanner, hvor DS/EN ISO 14971:2012 er blevet fuldt, for at vurdere risikoniveuet for Automatisk Ultralydsscanner som er færdigt produkt. 

Nedenfor er de identificerede risici indtegnet i en risikomatrix, hvilket gør det let at overskue, hvilke risici som skal reduceres, samt Automatisk Ultralydsscanners samlede risikoniveau.  

Tabel \ref{Niveau} viser risikomatrixens farvers betydning. 

\begin{figure}[H]
    \centering
    \includegraphics[width=0.30\textwidth]{figurer/r/Niveau}
    \caption{Risikomatrixens farvers betydning}
    \label{Niveau}
\end{figure}

Tabel \ref{Risiko} viser Automatisk Ultralydsscanners samlede risikoniveau. 

\begin{figure}[H]
    \centering
    \includegraphics[width=1\textwidth]{figurer/r/Risiko}
    \caption{Risikoniveau}
    \label{Risiko}
\end{figure}

ISO 14971:2012 specificere ikke hvad en acceptabel risiko er, men producenten er altid forpligtiget til at reducere risici mest muligt. Ud fra risikomatrixen ligger Automatisk Ultralydsscanners samlede risikoniveau, i den acceptable ende, da der er flest risici i det grønne område. Risiko R35 – \textit{Kalibrering mellem robotarm og 3D kamera er forkert} og R36 - \textit{Bevægelsesmønster af robotarm er uhensigtsmæssigt}, ligger i det uacceptable niveau. Derfor bør der laves risikoreduktion på disse to risici, hvor muligheden for at mindske sandsynligheden for at risiciene vil opstå vurderes. Dette kunne f. eks ske ved at lave hyppige tests af systemet, oplæring af Operatør i hvordan Automatisk Ultralydsscanner kalibreres, samt en detaljeret beskrivelse af hvad et hensigtsmæssigt bevægelsesmønster, for Automatisk Ultralydsscanner, er. 

\section{Softwaregodkendelse}
Da Automatisk Ultralydsscanner har software, som styrer robotarmen, skal krav til medicinsk software også overholdes. Der er krav om en dokumenteret udviklingsproces, vedligeholdelsesplan, risikohåndtering og plan for løsning af softwarefejl. Standarden DS/EN 63204:2006 - Software for medicinsk udstyr - Livscyklusprocesser for software \cite{software} er anvendt, til at sikre overensstemmelse med lovgivningen.

Den fulde medicinske godkendelse kan ses i bilag \ref {Godkendelsesprocedure} om Godkendelsesprocedure.
