\chapter{Analyser}\label{Analyser}

\section{Brugerundersøgelse}
Der er foretaget to slags brugerundersøgelse, en kvantitativ med potentielle patienter og to kvalitative interviews med en overlæge og en specialeansvarlig radiograf. Dette er medtaget for at belyse, hvordan vil en automatiseret ultralydsscanner til screening for brystkræft modtages af patienter og personale. 

\subsection{Spørgeskemaundersøgelse}
Undersøgelsen bestod af et kvalitativt spørgeskema med tre spørgsmål, omhandlende scanning med en automatiseret robotarm fremfor en læge, samt hvilke problemstillinger og fordele respondenterne ser ved en automatiseret robotarm. Spørgeskemaundersøgelsen blev lavet før projektet var færdigdefineret og derfor falder den lidt ved siden af projektet, den er dog stadig medtaget i projektet, fordi den kan give et billede af, hvordan patienter vil tage imod Automatisk Ultralydsscanner.  

Der var i alt 72 respondenter på spørgeskemaet, hvor størstedelen, 87,5\%, af respondenterne var positive for at blive scannet af robotarmen, hvis kvaliteten og sikkerhed er på højde med, hvad den er, når en læge foretager scanningen. De sidste 12,5\% som var negative for automatisk scanning med en robotarmen frygter, at robotarmen vil lave fejl, det bliver upersonligt, og at det vil give en fornemmelse af, at lægen har berøringsangst for patienterne. 

De problemstillinger respondenterne ser ved automatiserede ultralydsscanninger, var at robottens følsomhed mangler, og det måske kan gøre undersøgelsen ubehagelig og utryg for patienten. Samtidig nævner flere bekymringer for robottens evne til at scanne forskellige kropstyper.

Fordelene, som respondenterne så ved automatiserede ultralydsscanninger var, at robotten måske kan give økonomisk mening med kortere ventetider og spare tid og dermed frigøre ressourcer i form af personale til andre opgaver. Flere af respondenter mente, at en robotarm kan reproducere scanningerne og er derfor ikke afhængig af, hvor god lægen er. Den yngre del af respondenterne nævner ergonomiske fordele for lægen, mindre blufærdighed og langdistance-undersøgelser, som andre fordele. 

Se bilag \ref{Sporgeskemaundersogelse} for hele spørgeskemaundersøgelsen. 

\subsection{Besøg og interview på Aarhus Universitetshospital, Tage-Hansens Gade} 
Aarhus Universitetshospital, Tage-Hansens Gade blev kontaktet til inspiration og belysning af den daglige praksis på en røntgen- og skanningsafdeling, samt undersøgelse af sundhedsfagliges meninger om Automatisk Ultralydsscanner. Radiograf Tine Bisgaard indvilgede i at vise rundt på afdelingen samt svare på spørgsmål om afdelingens dagligdagen. På daværende tidspunkt var systemet ikke afgrænset til kun at omhandle scanninger og ikke kliniske scanninger. 

Tine Bisgaard vurdererede en mammografiscanning med røntgen af begge bryster til at tage 5 minutter, mens en ultralydsscanning blev vurderet til at tage omkring 10 minutter afhængigt af radiologens rutine. Tine Bisgaard mente ikke, at det vil være et problem at benytte en Automatisk Ultralydsscanner til at udføre scanninger, hvis man blot informerer patienterne. Hun ser dog en ulempe ved at lade en radiograf lave scanningerne idet, patienten ikke kan få svar med det samme, hvilket de normalt får når en radiolog udføre ultralydsscanningen. Tine Bisgaard nævnte yderligere en ulempe, som var tidselementet, hun vurderer, at det vil tage længere tid at foretage en scanning på en patient og derefter få en radiolog til at lokalisere knuder. Hun frygter at det vil tage lang tid for radiologen at vurdere billederne og forslår, at det kunne være en fordel, hvis Automatisk Ultralydsscanner selv kunne identificere knuder på videoclips fra ultralydsscanningen, så der samtidigt kunne sammenlignes med røntgenbillederne fra mammografiscanningen. 

Fordelene, Tine Bisgaard ser ved en Automatisk Ultralydsscanner, er, at man på afdelingerne er nødt til at tænke i nye baner i forhold til manglen på radiologer i Danmark. Derfor mener hun, at det vil være smart, hvis radiograferne kunne udføre en del af arbejdet med ultralyd for at spare tid og penge. Hende og en unavngiven kollega på afdelingen foreslår, at proceduren kan gøres simpel, dvs. gøre svare kvantitative f.eks. mål, ja/nej mm. Det ser hun som måden en radiograf vil kunne styre en Automatisk Ultralydsscanner.

På baggrund af interviewet og fremvisningen på afdelingen, blev der fokus på, hvordan en procedure for Automatisk Ultralydsscanner kan gøre simpel, og om substitueringen af radiologer med radiografer ville kunne være en økonomisk gevinst. 

Se bilag \ref{Tine} for hele interviewet med afdelingsradiograf Tine Bisgaard. 

\subsection{Interview med radiolog og ultralydsekspert}
Der blev foretaget et telefonisk interview og et opfølgende møde med radiolog og ultralydsekspert, Lars Bolvig, til belysning af proceduren ved ultralydsscanninger. Der var på forhånd defineret nogle spørgsmål angående lokalisering af knuder, hastigheder og tiden, en læge typisk vil bruge på en scanning og lokalisering af knuder. Ifølge Lars Bolvig vil en læge kunne lokalisere en knude i brystet på 2-3 minutter, mens hastigheden, der scannes med, er meget operatørafhængig. Det blev tydeliggjort, at Automatisk Ultralydsscanner vil kunne gøre mest gavn ved scanninger af brystet og ikke til kliniske scanninger. 

Ved scanninger af brystet, føres ultralydsproben i en sinus-lignende kurve med overlap, startende fra højre side af højre bryst til venstre side af højre bryst og derefter fra højre side af venstre bryst ud til venstre side af venstre bryst. Det skal sikres, at ultralydsproben starter og slutter uden for brystvævet, hvilket sikrer, at hele brystet er scannet. Bevægelsesmønsteret er illustreret i Figur \ref{Probensbevagelse} nedenfor. 

\begin{figure}[H]
    \centering
    \includegraphics[width=0.75\textwidth]{figurer/d/probebevagelse}
    \caption{Det specifikke bevægelsesmønster ved scanning}
    \label{Probensbevagelse}
\end{figure}

Lars Bolvig tilføjer også, at Automatisk Ultralydsscanner skal kunne betjenes af radiografer. Så Automatisk Ultralydsscanner fungerer ved, at radiografen tager billeder og sender videoclips af scanningen til radiologen, som diagnosticere og bestemmer videre behandlingsforløb.

Se bilag \ref{Telefoninterview} for hele telefoninterviewet med radiolog Lars Bolvig

\section{Økonomisk analyse}

Stadig bruge breakeven analysen til at se, hvis man indførte at mammografiscreening skulle udvides med en ultralydsscanning. Kunne man så spare penge ved at indføre Automatisk Ultralydsscanning. 
Find evidensbaseret litteraur omkring levere levetid pga. tidlig detektering af brystkræft, og om det har økonoiske fordele. 



Den økonomiske analyse er et overslag med udgangspunkt i en breakeven analyse. Analysen undersøger antallet af scanninger udført af radiografer og ikke radiologer, før udgifterne til Automatisk Ultralydsscanner. Efter interview med radiograf Tine Bisgaard og radiolog Lars Bolvig, blev det sandsynliggjort, at radiografer kan udføre simple procedurer, hvor videoclip af scanninger ses igennem af en radiolog, ligesom ved røntgen mammografi. Radiologer bruger i dag tid på at transportere sig til og fra scanningsstedet. Transporttid er derfor en vigtig variabel i breakeven analysen, der tager udgangspunkt i, hvor mange ressourcer der kan flyttes fra en radiolog til en radiograf, ift. omkostningen relateret til implementeringen af Automatiseret Ultralydsscanner. 

Det er antaget, at en radiologs gennemsnitlig løn er omkring 369 kr./timen, mens en radiografs gennemsnitlige løn er omkring 173 kr./timen \cite{Lon}. De samlede omkostning for anskaffelse af udstyret til opsætning af Automatisk Ultralydsscanner er fundet ved indsamling af priser fra forskellige hjemmesider (Se bilag \ref{Okonomi} Økonomi for flere oplysninger). De samlede faste omkostninger for fuld implementering inkluderer engangsudgifter for opsætning og oplæring af radiografer i anvendelse af Automatisk Ultralydsscanner. Den samlede pris for Automatisk Ultralydsscanner ligger omkring 219.305,64 kroner, hvor de samlede udgifter kan ses i Tabel \ref{FasteOmkostninger}. 

\begin{table}[htb]
\centering
\begin{tabular}{ | l | l | p{1\textwidth} | }
\hline
\textbf{Beskrivelse af udgift} & \textbf{DKK} \\\hline
Engangsudgifter til afskrivning & 201.668,00 \\\hline
Opsætning & 10.759,00 \\\hline
Oplæring af radiografer & 6.768,00 \\\hline
I alt & 219.305,64 \\\hline
\end{tabular}
\caption{Samlede udgifter for Automatisk Ultralydsscanner}
\label{FasteOmkostninger}
\end{table}

Der er for begge metoder blevet beregnet en pris for én ultralydsscreening. Prisen pr. ultralydsscreening med Automatiseret Ultralydsscanner er nogenlunde konstant, mens den varierer ved den traditionelle metode grundet forskellige transporttider for radiologen.  
Prisen pr. ultralydsscreening med Automatisk Ultralydsscanner er udregnet ved at antage, at det er en radiograf, der foretager forberedelse, 3D scanning og selve ultralydsscreeningen. En radiolog vil derefter bruge omkring 10 minutter på at tjekke billedet igennem for at se om patienten skal til en yderligere scanning. Prisen for dette er beregnet til 110,64 kroner. 

Prisen pr. ultralydsscreening med den traditionelle metode er udregnet ved at antage, at det er en radiolog, der foretager både forberedelse, ultralydsscreening og har en transporttid for at komme frem og tilbage. Hvis transport for radiologen er under fire minutter, er den nuværende metode billigst, hvorimod analysen viser, at ved transport længere end fire minutter er prisen pr. screening billigere med Automatiseret Ultralydsscanner.  
Tabel \ref{Breakeven} beskriver transportminutter, pris for én ultralydsscreening med den traditionelle metode, og antal screeninger med samme transporttid, før Automatisk Ultralydsscanner er betalt af. 

\begin{table}[htb]
\centering
\begin{tabular}{ | c | c | c | p{0.49\textwidth} | }
\hline
\textbf{Transporttid (min)} & \textbf{Pris for én screening (DKK)} & \textbf{Antal screeninger} \\\hline
5 & 116,89 & 35.052,51 \\\hline
8 & 135,34 & 8.873,09\\\hline
10 & 147,64 & 5.923,66\\\hline
15 & 178,39 & 3.235,19 \\\hline
20 & 209,14 & 2.225,25\\\hline
30 & 270,64 & 1.369,94\\\hline
45 & 362,89 & 868,95 \\\hline
60 & 455,14 & 636,26 \\\hline
\end{tabular}
\caption{Breakeven analyse for antal transportminutter}
\label{Breakeven}
\end{table}

Årligt foretages omkring 23.000 ultralydsundersøgelse af brystet i Region Midtjylland, og omkring 127.000 i hele landet. Det har ikke været muligt at finde præcise tal på, hvor mange af disse undersøgelser er scanninger eller screeninger. Hvis det antages, at bare 10 \% af undersøgelserne er screeninger, kan en Automatisk Ultralydsscanner potentielt kunne tjenes hjem på ét år, hvis radiologen skal bruge 20 minutter på transport til hver undersøgelse. 

Analysen er et overslag og ikke en nøje udført business case. Analysen ville fremstå bedre, hvis der havde medvirket flere radiologer og radiografer til estimering af tider. Det er generelt forsøgt at prissætte udgifterne forbundet med indførslen af Automatisk Ultralydsscanner relativt højt for at undgå for mange uforudsete omkostninger. Hvis et hospital vil købe udstyret, vil priserne for opsætning måske være lavere, hvis man laver en indkøbsaftale. Beregningerne har ikke taget højde for flere ultralydsscreeninger for en transporttid. 

Se bilag \ref{Okonomi} for hele den økonomiske analyse. 
