\chapter{Resumé}
\textbf{Baggrund} \newline
I Danmark tilbydes alle kvinder i alderen 50 til 69 år en rutinemæssig mammografiscreening. Mammografi med røntgen kan være uhensigtsmæssig at anvende, da det for kvinder med meget kirtelvæv, er svært at skelne kirtelvæv fra kræftknuder. Derfor er man i nogle tilfælde nødt til at supplere med en ultralydsscanning \cite{Ultralyd}.

Mammografi foretages i dag af radiografer eller røntgensygeplejersker, hvorefter røntgenbillederne sendes til en radiolog, som afgør om der skal foretages yderligere undersøgelser. I fremtiden kunne man forestille sig, at automatiserede ultralydsscanninger til screening for brystkræft kunne foretages med samme arbejdsprocedure, som man i dag udfører mammografi.

\textbf{Metoder} \newline
I udviklingsprocessen er Scrum brugt til at organisere projektet. Til undersøgelse af hvilke tiltag der skal til for at realisere Automatisk Ultralydsscanner er der lavet en spørgeskemaundersøgelse, en medicinsk godkendelse og en økonomiske analyse samt litteratursøgning. Til design af Automatisk Ultralydsscanner er UML og SysML anvendt. 

\textbf{Resultat} \newline
Der er udviklet et system kaldt Automatisk Ultralydsscanner. Automatisk Ultralydsscanner består af en PC applikation, som gør det muligt for et 3D kamera at konstruere en 3D model af et bryst, som en robotarm efterfølgende kan bevæge sig efter. Det lykkedes ikke at få robotarmen til at følge et bestemt bevægelsesmønster der skal til, for at lave en ultralydsscanning. 

\textbf{Diskussion} \newline


\textbf{Konklusion} \newline
Der blev udviklet et system, som til dels opfylder kravspecifikationen ift. afgrænsning af systemet. Mulighederne for medicinsk godkendelse af Automatisk Ultralydsscanner er blevet undersøgt - men systemet er ikke medicinsk godkendt. Tilføjelse af ultralydsscanninger til screeningsprogrammet er sandsynligvis ikke omkostningseffektivt.