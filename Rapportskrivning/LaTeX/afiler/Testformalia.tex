\chapter{Testformalia}\label{Testformlia}

\section{Godkendelseskriterier}
Godkendelsen af systemtesten består af to trin: 
\newline
1. Godkendelse af det overordnede dokument Accepttest. Dette gøres under afsnittet Testresultat, resultatet markeres med X og underskrives under "Godkendt af:".
\newline
2. Godkendelse af de enkelte dele i accepttesten. De enkelte dele i accepttesten godkendes, når testene af funktionelle og ikke-funktionelle krav er gennemført step for step og med resultater
i overensstemmelse med de forventede resultater.

\section{Testprocedure}
De funktionelle og ikke-funktionelle krav vil blive testet som beskrevet under hver test. I feltet ”resultat” markerer testpersonen 'Godkendt' eller 'Ikke godkendt' ud for det enkelte teststep. 
Godkendt betyder fejlfri gennemførsel. 
Ikke godkendt betyder, at teststeppet ikke kan gennemføres og godkendes. De fejl, der fører til, at steppet ikke kan gennemføres bliver beskrevet i et bilag til accepttesten, hvori fejlen bliver nærmere beskrevet. 

For at kunne gennemføre accepttest af funktionelle og ikke-funktionelle krav er det vigtigt, at systemet er stillet korrekt op, og at opstillingen kan stilles op på samme måde igen. Der refereres til opstillingerne i bilaget 'Sætningsliste'. 

\section{Forsøgsopstilling}
For at kunne reproducere testen, er det valgt ikke at benytte patienter, i stedet er et testobjekt, som er udformet som et bryst blevet anvendt, som erstatning for aktøren Patient. Der bliver benyttet en ultralydsdummy for at markere ultralydsprobens bane.  Testobjektet er beklædt med ler, hvor ultralydsdummyen, bestående af en pind, kan markere banen i leret.  

Under hver test, er forsøgsopstillingen og de pågældende aktører blevet beskrevet, som er en forudsætning for, at testene kan gennemføres. 

Aktøren Testperson agerer i denne accepttest Operatør, hvilket betder Testperson betjener Automatisk Ultralydsscanner uden at have forudsætningen om at kende til ultralyd. 