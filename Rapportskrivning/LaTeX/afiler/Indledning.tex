\chapter{Indledning}\label{Indledning}
Formålet med en accepttest er at få testet de funktionelle, use cases, og ikke-funktionelle krav beskrevet i bilag \ref{Kravspecifikation} Kravspecifikation. 

Accepttesten udføres typisk overfor kunden og er med til at sikre, at det færdige produkt lever op til kundens krav. Forkortelser og forklaring på forsøgsopstillingen, kan findes i bilag \ref{Satningsliste}  Sætningsliste. Sætningslisten skal være tilgængelig under udførslen af accepttesten, som opslagsværk for systemets standard positurer. 