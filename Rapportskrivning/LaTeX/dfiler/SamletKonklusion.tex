\chapter{Konklusion}\label{Konklusion}
Udviklingsprocessen har været vigtig for projektarbejdet. Gennem projektarbejdet er der anvendt forskellige redskaber til at opnå det bedst mulige resultat. 

Gruppen havde i første omgang i sinde at planlægge projektarbejde efter en sekventiel model, men besluttede hurtigt at dette ikke ville fungere, da gruppen i stedet ønskede at benytte en agil arbejdsmetode. Denne beslutning blev taget på baggrund af projektets mange ukendte variabler. Gruppen er tilfreds med at have benyttet elementer af Scrum som projektets udviklingsramme, og anvendelsen af de overordnede faser i V-modellen som rettesnor.

Anvendte Scrum-elementer, blev valgt ud fra, hvad der ville give værdi for udviklingsprocessen. På grund af gruppens størrelse blev der fravalgt at have en Scrum-master. Opdeling i sprints med task-boards og oprettelsen af burn-down charts blev anvendt, hvilket har givet god mening for gruppen. Gennem processen er gruppen med tiden blevet bedre til at anvende Scrum. Ved sprint 5 begyndte gruppen at bruge burn-down charts, hvilket gav en god indikator for, hvor god eller dårlig estimeringerne i sprintet var. Gruppen kunne dagligt sige noget om, hvor godt den var med i sprintet, grundet burn-down charts. Trello som Scrumboard har fungeret godt, og det samme gælder brugen af Git som versionsstyringsværktøj.

Risikohåndtering af projektarbejdet har ikke fungeret optimalt for gruppen til prioritering af projektets tasks. Projektets problemstillinger og arbejdsopgaver kunne ikke defineres på forhånd, grundet at projektet har været Proof of Concept. Dog lykkedes det gruppen at anvende risikohåndteringen i et begrænset omfang til prioritering af nogle opgaver i softwareudviklingen, hvilket fungerede udmærket. 

Selve gruppesammensætningen har været god, da gruppens medlemmer har suppleret hinanden godt, og arbejdsfordelingen har været fordelt alt efter det enkelte medlems kompetence- og interesseområde. Gruppen startede samarbejdet med at lave en samarbejdsaftale, som hovedsageligt blev brugt til forventningsafstemning. Aftalen kan være årsag til, at der ingen væsentlige konflikter er opstået i gruppen. 