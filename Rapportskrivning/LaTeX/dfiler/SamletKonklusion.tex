\chapter{Samlet Konklusion}\label{SamletKonklusion}
I dette afsnit er der lavet en samlet konklusion for, hvad der lykkedes, og hvad der ikke har fungeret i udviklingen af en Automatisk Ultralydsscanner. Udviklingsprocessen har været vigtig for projektarbejdet, og gennem projektarbejdet er der anvendt mange forskellige redskaber til at opnå det bedste mulige resultat. 

Gruppen havde planlagt at anvende V-modellen, men V-modellen fungerede ikke, da gruppen havde brug for en agil arbejdsmetode, hvor man hele tiden kunne vendte tilbage til opgaver, som afhæng af at andre opgaver var løst. Derfor er gruppen godt tilfreds med at have benyttet Scrum men også anvende de overordnede faser i v-modellen som rettesnor.

Ikke alle elementer af Scrum blev anvendt, og det blev valgt ud fra, hvad der ville give værdi for udviklingsprocessen. Opdeling i sprints, daglige Scrum møder og scrum-board blev anvendt, hvilket har givet meget værdi for gruppen, og gennem processen er gruppen med tiden blevet bedre til at anvende Scrum. Ved sprint 5 begyndte gruppe at bruge burn-down charts, hvilket gav et godt overblik over, hvor langt gruppen var i sprintet. Trello som Scrumboard har fungeret godt, og det samme gælder brugen af Git, med Source Tree som interface, til versionsstyring af dokumentationen. 

Noget, der ikke har fungeret så godt for gruppen, var risikovurdering af projektarbejdet til at prioritere projektets opgaver. Risikovurderingen var svær at anvende, da projektet er et udviklingsprojekt og projektets problemstillinger og arbejdsopgaver derfor ikke kunne defineres på forhånd. Dog lykkedes det gruppen at anvende risikovurdering i et begrænset omfang til prioritering af nogle opgaver i softwareudviklingen, hvilket fungerede godt. 

Selve gruppesammensætningen har været rigtig god, da gruppens medlemmer har suppleret hinanden godt og arbejdsfordelingen har været meget naturligt fordelt alt efter det enkelte medlems kompetence- og interesseområde. Gruppen startede samarbejdet med at lave en samarbejdsaftale, som hovedsageligt blev brugt til forventningsafstemning, og den kan være årsag til, at der ikke er opstået nogle konflikter i gruppen. 
