\chapter{Systemdesign}\label{Systemdesign}
Der er udarbejdet forskellige diagrammer på baggrund af de specificerede systemkrav.  Diagrammerne har til formål at dele systemet op i realiserbare dele for at vise designet af systemet. 

\section{Klassediagram} \label{Klassediagram}
Dette afsnit beskriver klasserne fra pakkediagrammet. Klassediagrammerne viser strukturen i systemet og deres relationer. Hver klasse indeholder de vigtigste metoder og attributer i klassen, der udgør funktionaliteten i PC Applikation. 

\subsection{GUI}
Denne klasse indeholder brugergrænsefladen af PC Applikation.

\let\labelitemi\labelitemii
\begin{itemize}
\item{\textbf{MainWindow}}\newline
Giver anledning til at foretage et 3D scan. Såfremt en 3D scanning er gennemført, giver det også anledning til at starte en ultralydsscanning.
Når menuen startes, oprettes en instans af RoboMaster, for at sætte Robotarm i standard positur. Dette er nødvendigt, hvis Robotarm skulle være i vejen for en 3D scanning.
Hvis der ikke er nogen forbindelse til Robotarm vil der komme en prompt med en besked om dette. 

\item{\textbf{3DScanMenu}}\newline
I denne menu er der mulighed for at se det nuværende dybdebillede, afgrænse området der skal 3D scannes og foretage en 3D scanning.

\item{\textbf{UltrasoundScanMenu}}\newline
I denne menu kan den procentvise færdiggørelse af ultralydsscanningen følges. Der er også mulighed for at pause samt afbryde ultralydsscanningsprocessen.
\end{itemize}

\begin{figure}[H]
    \centering
    \includegraphics[width=1\textwidth]{figurer/d/Design/Class/uml_class_gui}
    \caption{Klassediagram for GUI}
    \label{class_gui}
\end{figure}
\newpage

\subsection{ComputerVisionLibrary} \label{ComputerVisionLibrary}
Formålet med dette bibliotek er at få en afgrænset 3D scanning fra et 3D kamera.

\begin{itemize}
\item{\textbf{KinectFusionizeren}}\newline
Har til ansvar at åbne Kinect-sensoren, tage det nuværende dybdebillede fra sensoren og konvertere det til en mesh.

\item{\textbf{ComputerVisionMaster}}\newline
Denne klasse virker som den logiske grænseflade til KinectFusionizeren, hvor instansen af den nuværende mesh lagres her.
Andre klasser kan subscribe til ComputerVisionMasteren for at høre, hvornår der er en ny mesh tilgængelig.

\item{\textbf{Slicer}}\newline
Denne klasse sørger for at fjerne de punkter i en mesh, der er uinteressante:
\begin{enumerate}
\item{Faces der peger ned af, dvs fejlpunkter. Da 3D kamera er monteret i loftet, vil den ikke kunne se undersiden af det den scanner.}
\item{Duplikerede punkter. KinectFusionizer outputter punkter der er ens. Disse fjernes af optimeringsårsager.}
\item{Punkter og faces der er uden for det område der ønskes scannet. Dette inkluderer nærtliggende objekter som fx en væg, eller områder på patienten der ikke ønskes scannet.}
\end{enumerate}
\end{itemize}

\begin{figure}[H]
    \centering
    \includegraphics[width=1\textwidth]{figurer/d/Design/Class/uml_class_computervisionlibrary}
    \caption{Klassediagram for ComputerVisionLibrary}
    \label{class_ComputerVisionLib}
\end{figure}
\newpage

\subsection{CalculationLibrary}{\label {calculationlibraryafsnit}
Dette bibliotek agerer som bindeledet mellem ComputerVisionLibrary og RoboLibrary.

\begin{itemize}
\item{\textbf{CameraToRobotCalibrator}} \newline
Sørger for at konvertere 3D scanningen givet fra ComputerVisionLibrary fra 3D Kameras rum til Robotarms rum.
Dette sker ved en kæde af matrix-transformationer i en speciel rækkefølge. 
Normaltvis har man en translation, rotation og skalering, men da Robotarms og 3D Kameras koordinatsystemer begge er angivet i millimeter, er skaleringen unødvendig.
I tilfældet for dette projekt sker der først en rotering og derefter en translation, for at bestemme transformationsmatricen. 
Hvert punkt i en mesh konverteres så til det nye space. Se reference \cite{RumTransformation} for inspirationen til denne klasse.

\item{\textbf{Smoother}}\newline
Denne klasse har til ansvar at udjævne en mesh. Med udjævning forstås, at ensrette retningsvektorerne i en mesh's faces. Dette vil bivirke at ved en scanning af et relativt flat objekt og N antal smoothing-runs, vil meshens faces være stort set ens.
Dette er nødvendigt da 3D kameras output kan indeholde anomalier og dermed vil nogle normaler på 3D scanningens overflade være ekstreme/deforme. Med ekstreme normaler menes der normaler der ikke ligner dens nærmeste nabo-normaler på meshen. Disse anomalier kan bivirke at Robotarm roteres tilsyneladende forkert imod meshen. Udjævningen sker gennem laplacian smoothing hvor alle retningsvektorer i overfladen ensrettes. Se \cite{Smooth} for forklaring af anvendt algoritme.

\item{\textbf{PathCreator}}\newline
Klassen afgør listen af punkter i en mesh som der skal findes positurer til Robotarm ud fra.
For at afgøre stien genereres der en 'bølge' - i implementeringen en squarewave - af punkter der draves over meshen.
De vertices i meshen der tilnærmer sig punkterne i bølgen bedst vil blive udvalgt til stien.

\item{\textbf{RobotPoseCreator}}\newline
I denne klasse vil konverteringen af en mesh-sti til en liste af positurer ske.
For hvert punkt i mesh-stien, vil en vertex' normal findes. 
Ved hjælp af normalen, sti-punktets koordinater samt længden på Robotarms probe kan den forskudte Robotarm position findes.
Inverteres denne normal, kan det ses som en retningsvektor for en Robotarm.
Retningsvektoren konverteres først til en roll, pitch og yaw - altså roteringer omkring de tre retningsakser; X, Y og Z.
Da man ikke kan afgøre alle tre værdier ud fra en retningsvektor alene, sættes pitch til 0, da det er muligt at 'pege' et vilkårligt sted med en roll-rotering og en yaw-rotering. Disse værdier konverteres herefter til en rotationsvektor.
Positionsvektoren og rotationsvektoren udgør til sammen en positur, som tilføjes til listen af positurer. Matematikken for udregningen af rotationen kan ses på næste side.
\newpage

\subsubsection{RobotPoseCreator rotationsmatematik}{\label {rotationsmatematik}}
Givet en tredimensionel retningsvektor med en længde på 1
$$
v_{direction} = (X, Y, Z)
$$
Find de tre rotationer om de tre forskellige akser. Med en retningsvektor alene kan én af disse rotationer ikke findes. Da man vil kunne pege i en vilkårlig retning med en roll-rotering og en yaw-rotering, er pitch sat til 0.
$$ 
roll = acos(Z) \qquad 
pitch = 0 \qquad 
yaw = \begin{cases} 
	-acos(-Y) & X \leq 0\\
	acos(-Y)  & X > 0 \\
\end{cases}
$$

Opstil matricerne der skal bruges for at konvertere roll, pitch og yaw til en rotationsvektor.
%%%%%%%%%%%%%%%%%%%%
% Matricerne 
$$ 
Roll_M = \begin{bmatrix}
    1 & 0 & 0 \\
    0 & cos(roll) & -sin(roll) \\
    0 & sin(roll) & cos(roll)
\end{bmatrix}
\quad
Pitch_M = \begin{bmatrix}
    cos(pitch) & 0 & sin(pitch) \\
    0 & 1 & 0 \\
    -sin(pitch) & 0 & cos(pitch)
\end{bmatrix} 
$$
$$
Yaw_M = \begin{bmatrix}
    cos(yaw) & -sin(yaw) & 0 \\
    sin(yaw) & cos(yaw) & 0 \\
    0 & 0 & 1
\end{bmatrix}
$$ 

Ordnen af rotationerne er vigtig, derfor vil der først foregå en roll rotering, en pitch rotering og til sidst en yaw rotering. Prikken mellem matricerne her er dot-produktet.
$$\mathbb{R} = Yaw_M \cdot Pitch_M \cdot Roll_M $$

Dernæst kan $\theta$ og $\mu$ findes, der bruges til beregningen af $r_x$, $r_y$ samt $r_z$, der samlet udgør den endelige rotationsvektor $v_{rotation}$.

$$\theta = acos\bigg(\frac{\mathbb{R}_{0,0}+\mathbb{R}_{1,1}+\mathbb{R}_{2,2}-1}{2}\bigg) \qquad \mu = \frac{1}{2 \times sin(\theta)}$$

$$r_x =\mu \times(\mathbb{R}_{2,1}-\mathbb{R}_{1,2}) \times\theta $$
$$r_y =\mu \times(\mathbb{R}_{0,2}-\mathbb{R}_{2,0}) \times\theta $$
$$r_z =\mu \times(\mathbb{R}_{1,0}-\mathbb{R}_{0,1}) \times\theta $$
$$v_{rotation} = (r_x, r_y, r_z)$$

\end{itemize}

\begin{figure}[H]
    \centering
    \includegraphics[width=1\textwidth]{figurer/d/Design/Class/uml_class_calculationlibrary}
    \caption{Klassediagram for CalculationLibrary}
    \label{class_ConversionLib}
\end{figure}
\newpage

\subsection{RoboLibrary}
Biblioteket giver mulighed for kommunikation med Robotarm.

\begin{itemize}
\item{\textbf{RoboMaster}}\newline
Klassen agerer som bindeled mellem de øvrige klasser i biblioteket og GUI.

\item{\textbf{PathFeeder}} \newline
Står for at gennemløbe hver positur i listen, og kommunikere med Data for at finde ud af hvornår den næste positur skal sendes til Robotarm.

\item{\textbf{Data}}\newline
Klassen virker som en grænseflade mellem den 'logiske' del af biblioteket og dens underliggende reader/writer klasser.

\item{\textbf{Reader}}\newline
Denne klasse står for kontinuerligt at læse data fra Robotarm, for at afgøre dens nuværende positur. Denne klasse er kopieret og ændret fra det tidligere bachelorprojekt, TRU.

\item{\textbf{Analyzer}} \newline
Klassen konverterer det indlæste data til en objekt-orienteret model, altså transformation af bytes til Robotarms nuværende positur. Denne klasse er kopieret og ændret fra det tidligere bachelorprojekt, TRU.

\item{\textbf{Writer}}\newline
Klassen har til ansvar at omskrive værdier til binær data. Den omskriver både positurer samt konfigurationer. Denne klasse er kopieret og ændret fra det tidligere bachelorprojekt, TRU.

\item{\textbf{Modbus}}\newline
Denne klasse skriver binær data ud på Robotarms IP gennem modbus-protokollen. Denne klasse er kopieret og ændret fra det tidligere bachelorprojekt, TRU.
\end{itemize}

\begin{figure}[H]
    \centering
    \includegraphics[width=1\textwidth]{figurer/d/Design/Class/uml_class_robolibrary}
    \caption{Klassediagram for RoboLibrary}
    \label{class_RoboLib}
\end{figure}
\newpage

\section{Sekvensdiagrammer}
Der er på baggrund af klassediagrammerne lavet sekvensdiagrammer, som beskriver systemets funktionalitet, og hvor de vigtigste metoder og attributter imellem klasserne er identificeret.

Nedenstående sektioner vil beskrive de vigtigste sekvensdiagrammer i Automatisk Ultralydsscanner og fremvise, hvordan klasserne indbyrdes kommunikerer. 
Bemærk at for Read Robot Data samt Feed Path er Access Point undladt, da den blot videregiver informationen til Robotarm.

\subsection{Read Robot Data} 
Reader initieres med en IP, hvor den skal lytte på. 
Der åbnes en socket på denne IP, og derefter lytter den kontinuert i en baggrundstråd. 
Readeren giver de rå data videre til Analyzer som konverterer dem til Robotarms nuværende positur.

\begin{figure}[H]
    \centering
    \includegraphics[width=1\textwidth]{figurer/d/Design/Sequence/sd_reading}
    \caption{Sekvensdiagram for Read Robot Data - indlæsning af Robotarms positur}
    \label{sd_reading}
\end{figure}

\subsection{3D scan}
Ved 3D scan vil KinectFusionizeren stå for at få en 3D scanning fra Kinect-sensoren. I denne klasse sørges der for at konvertere et dybdebillede (en depth-frame) fra Kinect's infrarøde kamera til en point cloud. Der afventes det næste dybdebillede, hvis point cloud integreres med det forrige billede. Når der er tilstrækkelige point clouds skal det konverteres til en 3D model.  Disse point clouds trianguleres, så der fås en mesh der efterfølgende kan bearbejdes. 3DScanMenu modtager meshen gennem event-kommunikation.
Gennem de beskræringsparametre der er valgt i menuen vil meshen dernæst beskæres i klassen Slicer, så kun det ønskede område fremkommer. Efter beskæringen vises meshen som en rotérbar 3D model i menuen.

\begin{figure}[H]
    \centering
    \includegraphics[width=1\textwidth]{figurer/d/Design/Sequence/sd_3Dscan}
    \caption{Sekvensdiagram for 3D scan}
    \label{sd_3Dscan}
\end{figure}
\newpage

\subsection{Feed Path}
Efter konvertering af mesh output til robot positur sti, vil listen af positurer sendes fra RoboMaster til PathFeeder. 
Se sekvensdiagrammet 'Path Creation' for en gennemgang af hvordan disse positurer findes.
PathFeeder gennemløber hver positur og sender den næste i listen til Data, som videregiver posituren til Writer.
RoboMaster informeres løbende om hvor langt PathFeeder er med at gennemløbe punkter.
Writer konverterer posituren til binær data, og ModBus skriver dataen ud på Robotarms register.
Hernæst ser PathFeeder på om Robotarms nuværende positur har nærmet sig den ønskede positur. 
Når den er tæt nok på, hoppes der ud af 'while'-løkken, og den næste positur kan sendes.
Den Alt der er her skal forstås som at PathFeeder kører i sin egen baggrundstråd der kan pauses. 
Ved terminering af denne baggrundstråd vil PathFeeder stoppe med at videregive nye positurer til Robotarm.

\begin{figure}[H]
    \centering
    \includegraphics[width=1\textwidth] {figurer/d/Design/Sequence/sd_feedpath}
    \caption{Sekvensdiagram for Feed Path}
    \label{sd_feedpath}
\end{figure}

\subsection{Pathcreation}
Ved tryk på Ultralydsscan-knappen sendes den scannede mesh videre til CalculationMaster. Se sekvensdiagrammet '3D scan' for at se hvordan den scannede mesh opnås.
Først skal meshen konverteres fra 3D Kameras rum til Robotarms rum, dette sker i CameraToRobotCalibrator, hvor meshen roteres og translateres ift. hvor 3D Kamera befinder sig i virkeligheden.
Efter transformationen skal meshen udjævnes, så de rå og ekstreme normaler i meshen udglattes. Loopet afgør hvor mange gangen meshen skal gennemkøres filteret. Dernæst sendes den konverterede mesh til PathCreator så der oprettes en liste af de punkter i meshen vi ønsker at Robotarm skal gennemgå.
Til sidst skal stien der findes i PathCreator konverteres til Robotarm positurer, da stien i PathCreator kun fortæller position direkte på meshen. Med denne sti får Robotarm afdækket overfladen på Patient, hvis den har en Ultralydsscanner monteret.
Stien kan nu videresendes til RoboMaster - se sekvensdiagrammet 'Feed Path' Figur \ref{sd_feedpath}.

\begin{figure}[H]
    \centering
    \includegraphics[width=1.4\textwidth, angle =90]{figurer/d/Design/Sequence/sd_pathcreation}
    \caption{Sekvensdiagram for Pathcreation}
    \label{sd_pathcreation}
\end{figure}
\newpage

\section{Tilstandsdiagram}
Dette afsnit beskriver adfærden i systemet ved brug af et tilstandsdiagram. Tilstandsdiagrammet beskriver overgange mellem forskellige tilstande. I UC3: Ultralydsscan brystområde kan Operatør vælge at pause scanningen midlertidig og enten genoptage eller helt stoppe scanning. Figur \ref{stm_Ultra} beskriver Robotarms forskellige tilstande under udførelse af ultralydsscanning. 

\begin{figure}[H]
    \centering
    \includegraphics[width=0.7\textwidth]{figurer/d/Design/stm_UC3}
    \caption{Tilstandsdiagram for ultralydsscanning}
    \label{stm_Ultra}
\end{figure}



